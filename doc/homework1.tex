\documentclass[11pt,letterpaper]{article}
\usepackage{amsmath}
\usepackage[margin=.75in]{geometry}
\usepackage{amssymb}
\usepackage{graphicx}
\usepackage{float}
\usepackage{appendix}
\usepackage[utf8]{inputenc}
\usepackage{hyperref}
\floatstyle{ruled} \restylefloat{table} \restylefloat{figure}

\newcommand{\floatintro}[1]{
  
  \vspace*{0.1in}
  
  {\footnotesize

    #1
    
  }
  
  \vspace*{0.1in} }
\newcommand{\Hline}{\noindent\rule{17cm}{0.5pt}}
\title{Homework 1 : Computational Economics \\
Search Models}
\author{Dhananjay Ghei}
\date{October 31, 2018}
\begin{document}
\maketitle
\section*{Search Models and Labor Wedge}
\begin{enumerate}
\item First show that there exists an equilibrium in the following
transformed variables: $\tilde{c}(s^t) =
c(s^t)z(s^t)^{\frac{-1}{1-\alpha}}$, $\tilde{k}(s^t) =
k(s^t)z(s^t)^{\frac{-1}{1-\alpha}}$ and $\tilde{w}(s^t) = w(s^t)
z(s^t)^{\frac{-1}{1-\alpha}}$. Then show that there exists a Markovian
solution with $k,n,s$ as state variables. Boil down the equations
below to two functional equations in two unknown functions:
$\theta(k,n,s)$ and $\tilde{c}(k,n,s)$. \\ \Hline \\
Appendix \ref{appendix:a} gives the calculations for the equations
derived below with the appropriate transformations.\\
The Euler equation is given by:
\begin{align*}
  1 = \beta \sum_{s^{t+1}|s^t} \pi(s_{t+1}|s_t)
  \frac{\tilde{c}(s^t)}{\tilde{c}(s^{t+1})}
  e^{-\frac{s_{t+1}}{1-\alpha}} \bigg(\alpha \bigg(
  \frac{\tilde{k}(s^{t+1})}{n(s^{t+1}) - \theta(s^{t+1})
  (1-n(s^{t+1}))}\bigg)^{\alpha-1} + 1 - \delta \bigg)
\end{align*}
The condition w.r.t employment is given by:
\begin{align*}
  (1-\alpha) \bigg(\frac{\tilde{k}(s^t)}{n(s^t)-\theta(s^t)(1-n(s^t))}
  \bigg)^{\alpha} = \beta \mu(\theta(s^t)) \sum_{s^{t+1}|s^t}
  \pi(s_{t+1}|s_t)\\
   \times \bigg( (1-\alpha) \bigg(\frac{\tilde{k}(s^{t+1})}{n(s^{t+1})
  - \theta (s^{t+1}) (1-n(s^{t+1}))}\bigg)^{\alpha} \bigg(1 +
  \frac{1-x}{\mu(\theta (s^{t+1}))} \bigg) - \tilde{w}(s^{t+1}) \bigg)
\end{align*}
The wage equation evaluated in history $s^{t+1}$ reduces to:
\begin{align*}
  (1-\tau) \tilde{w}(s^{t+1}) = (1-\phi) \gamma \tilde{c}(s^{t+1})
  + \phi (1-\tau)(1-\alpha) \bigg(\frac{\tilde{k}(s^{t+1})}{n(s^{t+1})
  - \theta (s^{t+1})(1-n(s^{t+1}))} \bigg)^{\alpha} (1+\theta(s^{t+1}))
\end{align*}
Eliminating wage $\tilde{w}(s^{t+1})$ between the last two equations
gives:
\begin{align*}
  (1-\alpha) \bigg(\frac{\tilde{k}(s^t)}{n(s^t) -
  \theta(s^t)(1-n(s^t))} \bigg)^{\alpha} = \beta \mu(\theta(s^t))
  \sum_{s^{t+1}|s^t} \pi(s_{t+1}|s_t)
  \frac{\tilde{c}(s^t)}{\tilde{c}(s^{t+1})}\\ \bigg(-
  \frac{(1-\phi)\gamma \tilde{c}(s^{t+1})}{1-\tau} + (1-\alpha) \bigg(
  \frac{\tilde{k}(s^{t+1})}{n(s^{t+1}) - \theta(s^{t+1})(1-n(s^{t+1}))}
  \bigg)^{\alpha} \bigg(\frac{1-x}{\mu(\theta(s^{t+1}))} + 1 -\phi -
  \phi \theta(s^{t+1}) \bigg)\bigg)
\end{align*}
The resource constraint is given by:
\begin{align*}
  \tilde{k}(s^{t+1}) e^{\frac{s_{t+1}}{1-\alpha}} =
  (\tilde{k}(s^t))^{\alpha} (n(s^t) - \theta(s^t)(1-n(s^t)))^{1-\alpha}
  + (1-\delta) \tilde{k}(s^t) -\tilde{c}(s^t)
\end{align*}
The law of motion for employment is given by:
\begin{align*}
  n(s^{t+1}) = (1-x) n(s^t) + f(\theta(s^t))(1-n(s^t))
\end{align*}
In order to show that there exists a Markovian solution, we first
conjecture that the solution is given as constants $\bar{c}, \bar{k},
\bar{w}, \bar{w}, \bar{\theta}$. Then, we
plug in these values into the 5 equations above. If there is such an
equilibrium, the equations
are given by:
\begin{align*}
  1 &= \beta \sum \pi(s_{t+1}|s_t) e^{-\frac{s_{t+1}}{1-\alpha}}
  \bigg(\alpha \bigg(\frac{\bar{k}}{\bar{n} -
  \bar{\theta}(1-\bar{n})}\bigg)^{\alpha-1} + 1 - \delta\bigg)\\
  (1-\alpha)
  \bigg(\frac{\bar{k}}{\bar{n}-\bar{\theta}(1-\bar{n})}\bigg)^{\alpha}
  &= \beta \mu(\bar{\theta}) \sum \pi(s_{t+1}|s_t)
  \times \bigg((1-\alpha) \bigg(\frac{\bar{k}}{\bar n - \bar \theta
  (1-\bar n)}\bigg)^{\alpha} \bigg(1 + \frac{1-x}{\mu(\bar \theta)}
  \bigg) - \bar{w}\bigg)\\
  \bar n &= (1-x)\bar n + f(\bar \theta) (1-\bar n)\\
  \bar k e^{\frac{s_{t+1}}{1-\alpha}} &= \bar k^{\alpha} (\bar n - \bar
  \theta (1-\bar n))^{1-\alpha} + (1-\delta) \bar k - \bar c\\
  (1-\tau) \bar w &= (1-\phi) \gamma \bar c + \phi (1-\alpha) (1-\tau)
  \bigg(\frac{\bar k}{\bar n - \bar \theta (1-\bar n)} \bigg)^{\alpha} (1+\bar \theta)
\end{align*}
Substituting out $\bar c, \bar n, \bar k$, we get an equation in
$\bar \theta$ and $\bar w$ as:
% Add equation
There is an equilibrium of this form if and only if this equation can
be satisfied in any history $s^t$. Thus, this requires that there
exists a number $\bar s$ satisfying:
% Add equation

Finally, let the functions $\Theta$ and $C$ define the
recruiter-employment ratio and equilibrium consumption relative to
trend as functions of the current state $(s,n,\tilde{k})$ so,
$\theta(s^t) = \Theta(s_t, n(s^t), \tilde{k}(s^t))$ and
$\tilde{c}(s^t) = C(s_t, n(s^t), \tilde{k}(s^t))$. Substituting in the
expressions and eliminating $\tilde{k}(s^{t+1})$ and $n(s^{t+1})$, we
get two nonlinear equations in $\Theta$ and $C$. Appendix
\ref{appendix:b} gives the calculations for this transformation. The
functional equations are given as:

\item Compute the non stochastic steady state. Log linearise around
  the non stochastic steady state to get a system of linear
  expectational difference equations and solve for the policy
  rules. For example:
  \begin{align*}
    \text{log}\theta_{t+1} = \text{log}\bar{\theta} + \theta_s
    (s_{t+1}-\bar{s}) + \theta_n (\text{log}n_t - \text{log}\bar{n}) +
    \theta_k (\text{log}k_t - \text{log}\bar{k})
  \end{align*}
\Hline \\
Appendix \ref{appendix:c} uses the equations in part 1) to derive the
equations for the non stochastic steady state. The steady state
equations are given by:
\begin{align*}
  1 &= \beta e^{-\frac{\bar{s}}{1-\alpha}}
  \bigg(\alpha \bigg(\frac{\bar{k}}{\bar{n} -
  \bar{\theta}(1-\bar{n})}\bigg)^{\alpha-1} + 1 - \delta\bigg)\\
  (1-\alpha)
  \bigg(\frac{\bar{k}}{\bar{n}-\bar{\theta}(1-\bar{n})}\bigg)^{\alpha}
  &= \beta \mu(\bar{\theta}) \bigg((1-\alpha)
    \bigg(\frac{\bar{k}}{\bar n - \bar \theta (1-\bar
    n)}\bigg)^{\alpha} \bigg(1 + \frac{1-x}{\mu(\bar \theta)} \bigg) -
    \bar{w}\bigg)\\
    \bar n &= (1-x)\bar n + f(\bar \theta) (1-\bar n)\\
  \bar k e^{\frac{\bar{s}}{1-\alpha}} &= \bar k^{\alpha} (\bar n - \bar
  \theta (1-\bar n))^{1-\alpha} + (1-\delta) \bar k - \bar c\\
  (1-\tau) \bar w &= (1-\phi) \gamma \bar c + \phi (1-\alpha) (1-\tau)
  \bigg(\frac{\bar k}{\bar n - \bar \theta (1-\bar n)} \bigg)^{\alpha}
                    (1+\bar \theta) 
\end{align*}

We put these equations along with the parameter values to solve for
the steady state using a non-linear solver. The steady state is given
by: $\bar n=0.949$, $\bar k=218.239$, $\bar c=4.695$, $\bar w=4.016$,
$\bar \theta = 0.077$.\\
In order to log-linearise the equations, we convert all the variables
into log deviations from steady state and then use a first order
Taylor approximation of the equation around 0 to get the
coefficients. The log linearised equations are given by:
\begin{align*}
  \log k_{t+1} - \log \bar k &= -.610(s_t-\bar s) + .0186(\log n_t - \log
  \bar n) + 0.991(\log k_t - \log \bar k)\\
  \log n_{t+1} - \log \bar n &= 0.026(s_t-\bar s) + 0.312(\log n_t - \log
  \bar n) -0.047(\log k_t - \log \bar k)\\
  \log \tilde y_{t} - \log \bar y &= -0.003(s_t-\bar s) + 0.726(\log n_t - \log
  \bar n) + 0.337(\log k_t - \log \bar k)\\
  \log \tilde w_{t} - \log \bar w &= -0.236(s_t-\bar s) - 0.205(\log n_t - \log
  \bar n) + 0.345(\log k_t - \log \bar k)
\end{align*}
where the bar variables are the steady state values of these variables.
\item Using the calibration in Table 3.2 to compute the IRF and the
  ergodic moments of all the relevant variables. Please check your
  results against figure 3.2 in the book. \\ \Hline \\
For most of the parameters, Shimer uses the same calibration as in
Table 3.2. However, he changes some of the parameters. He sets
$\bar{s} = 0.0012$, $\rho = 0.4$ and $\zeta = 0.00325$. 
Table \ref{tab:calibration} gives the values
of the remaining parameters.
\begin{table}[htbp!]
\label{tab:calibration}
\caption{Calibration of parameters}
\floatintro{These are the same parameters as in Shimer's book from
  Table 3.2}
\centering
\begin{tabular}{lrr}
  \hline
  Parameter & Description & Values \\
  \hline
  $\beta$ & Discounting & 0.996 \\
  $\alpha$ & Share of capital & 0.33 \\
  $\tau$ & Tax rate & 0.4 \\
  $x$ & Employment exit probability & 0.034 \\
  $\phi$ & Worker's bargaining power & 0.5 \\
  $\delta$ & Depreciation & 0.0028 \\
  \hline
\end{tabular}
\end{table}
The log linear solutions for the relevant variables can be calculated
directly once we have the solutions for the main variables as given in
the previous part. As an example, the coefficients for the log linear
solution of consumption-output ratio $c/y$ is directly given by
subtracting the coefficients of output from that of consumption. We
can do that same exercise for the labor share of income $wn/y$. \\
Table \ref{tab:correlations} shows the ergodic moment of all the
relevant variables. In particular, one can compare this table directly
with Table 3.5 from Shimer's book. Note that, the relative standard
deviations and the correlations are close to the values in the book.

\begin{table}[htbp!]
  \floatintro{The table shows the co-movement of variables in an
    infinite sample. The table closely resembles to the one Shimer has
    in his book. The corresponding table to look at is Table 3.5 on Page
    101. The variables have similar ergodic moments as in the book.}
  \caption{Model with capital, stochastic trend. Co-movements of
    variables in infinite samples.}
  \label{tab:correlations}
  \centering
  \resizebox{\textwidth}{!}{
    \begin{tabular}{lr|rrrrrrrrr}
      &  & $\tilde{y}$ & $\tilde{c}$ & $\theta$ & $\tilde{k}$ & $n$ &
                                                                      $wn/y$ & $c/y$ & $\hat{\tau}$ & $s$ \\ \hline
      & Relative sd. & 1 & 2.09 & 9.632 & 3.475 & 0.239 & 0.129 & 1.1
                                                                                     & 1.291 & 0.541 \\ \hline
      & $\tilde{y}$ & 1 & 0.99468 & -0.99657 & 0.99997 & -0.99874 & 0.01101 & 0.98064 & -0.97011 & -0.06679 \\
      & $\tilde{c}$ & - & 1 & -0.98281 & 0.99514 & -0.99583 & 0.11375 & 0.9956 & -0.98992 & 0.03617 \\
      Correlations & $\theta$ & - & - & 1 & -0.99621 & 0.99391 & 0.07163 & -0.96122 & 0.94676 & 0.14895 \\
      & $\tilde{k}$ & - & - & - & 1 & -0.99911 & 0.0154 & 0.98156 & -0.97117 & -0.06238 \\
      & $n$ & - & - & - & - & 1 & -0.03118 & -0.98398 & 0.974 & 0.04636 \\
      & $wn/y$ & - & - & - & - & - & 1 & 0.20612 & -0.25329 & 0.99696 \\
      & $c/y$ & - & - & - & - & - & - & 1 & -0.99879 & 0.12946 \\
      & $\hat{\tau}$ & - & - & - & - & - & - & - & 1 & -0.17731 \\
      & $s$ & - & - & - & - & - & - & - & - & 1 \\ \hline
    \end{tabular}}
\end{table}
\textbf{TODO:} Impulse response functions. The impulse response graphs
do not match the figures from Shimer's book.
\item Derive an expression for labor wedge in this economy using the
  consumption output ratio and the hours which here is a fraction of
  HH that are employed. Obtain a log linear expansion of the wedge
  using the log linear policy rules. \\ \Hline \\
The expression for labor wedge is given by:
\begin{align*}
  \hat{\tau}(s^t) = 1 - \frac{\hat \gamma}{1-\alpha}(c(s^t)/y(s^t))n(s^t)
\end{align*}
Considering the fact that the true tax factor is 0.4,the disutility of
work is $\hat \gamma = .513$. Next, we log linearise the equation for
wedge and the policy rule for labor wedge is given by:
\begin{align*}
  \log \hat \tau_t = \log .4 -0.572(s-\bar s) -0.429 (\log n_t - \log \bar n) -0.396
  (\log k_t - \log \bar k)
\end{align*}
\item What is the correlation of the labor wedge with output and
  employment? You can sign it by staring at the policy rules. What is
  the ergodic std. of the labor wedge? How does the correlations and
  volatility compare to data. \\ \Hline \\
We can look at Table \ref{tab:correlations} and see that the
correlation of labor wedge with output is -0.9701 and the correlation with
employment is 0.974. The relative standard deviation of labor wedge in an
infinite sample is equal to 1.29. The labor wedge is countercyclical.
\item Now extract a TFP series such that the simulate path of $y(s^t)$
  is exactly as in the data. Start with the steady state level of
  capital and employment. Using the policy rule for output reverse
  engineer a shock such that detrended output matches exactly to the
  value in the data. Now update the state variables using your policy
  rules and keep iterating this procedure. How does the TFP series
  extracted compare to a Solow residual in the data? \\ \Hline \\ 
\end{enumerate}
\newpage
\begin{appendices}
  \section{Transforming the variables on a balanced growth
    path} \label{appendix:a}
  \section{Reducing the set of equations to a pair of nonlinear
    equations} \label{appendix:b}
  \section{Steady state calculations} \label{appendix:c}
\end{appendices}
\end{document}
